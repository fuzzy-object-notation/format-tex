\documentclass[UTF8]{article}
\usepackage[left=1in, right=1in, top=1in, bottom=1in]{geometry}

\usepackage{xeCJK}
\usepackage{fancyhdr}
\setCJKmainfont{WenQuanYi Micro Hei}
\setCJKmonofont{WenQuanYi Micro Hei Mono}

\pagestyle{fancy}
% \ExamHead{\today}

\fancyhead[L]{\today}
\fancyhead[C]{}
\fancyhead[R]{\textbf{FML with Web}}
\fancyfoot[L]{}
\fancyfoot[C]{\thepage}
\fancyfoot[R]{}
\renewcommand{\headrulewidth}{0.4pt}
\renewcommand{\footrulewidth}{0.4pt}

\let\ds\displaystyle

\title{FML with Web設計規格書}

\begin{document}
	
  \section{實踐目標}
		\subsection{基本功能}
			能顯示模糊決策函數的界面 \\
			能顯示決策關聯圖 \\
			GA實做 \\
			手動建立Ruler \\
			匯入建立Ruler \\
			手動建立FN \\
			匯入建立FN \\
			會出圖/PDF/json/csv \\
			專案能進行存檔與讀檔 \\

		\subsection{擴充功能}
			Vue.js表現一頁解決的界面 \\
			頁尾可連結GitHub \\
			捐款連結 \\
			手機板擴充 \\
			中/英文切換 \\
			FUZZY推論方法設計(EX: min / or / 減法 ...) \\
			公式:very ^2, more-or-less ^0.5
			推論的規則權重設計 \\
			FNN \\

  \section{使用規格}
		\subsection{JS模組相關}
		框架:Vue.js \\
		GPU支援:GPU.js \\
		圖表:D3.js/C3.js/vis.js \\
		封裝:parcel.js \\
		節點資訊表示:sigma.js \\
		特效:anime.js

		\subsection{UI相關}
		material-design-lite \\

\end{document}